%!TEX root = ./main.tex

\section{Local methods for design \& verification}\label{section:local_methods}

% Motivate this section by referencing the design analysis loop from the introduction.

% Specific context: when the designer knows a good starting point and just wants to tune it to be
% robust.

% Introduce a few motivating examples - box pushing, sensor placement, trajectory planning

\subsection{Problem statement}

% Math-ify the setting: free parameters, exogenous parameters, simulator, cost function. Maybe a table to summarize the notation.

% Show how that works for the motivating example

\subsection{Variance-regularized design optimization}

% Introduce VR optimization formulation

% Explain why we need autodiff here (expense of FD when we need multiple objective calls for regularization)

\subsubsection{Case study: multi-robot manipulation}

% Problem setup

% Research questions relevant to state of the art.

% Results

% How do our results compare to the state of the art?

\subsection{Local adversarial testing}

% Motivate adversarial optimization, link to adversarial training in ML

% Mathematical optimization formulation

% Theory (nash equilibrium)

\subsubsection{Case study: robust planning from formal specifications}

% Problem setup: satellite

% Research questions relevant to state of the art.

% Results

% How do our results compare to the state of the art?

\subsection{Discussion \& Limitations}

% Advantages of the differentiable simulation perspective

% Connection to how it can impact working engineers

% Limitations of the local perspective, set up the next section
