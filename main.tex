%---------------------------------------------------------------------------%
%								   Preamble									%
%---------------------------------------------------------------------------%
% declare document class, 12pt, lettersize and article 
% could also be report, however section headers turn into chapters
\documentclass[12pt, lettersize]{article}

% import preamble.sty for packages
% import refextdoc.sty for subfile crossreferencing
% note the relative import. Because subfiles (e.g. abstract, introduction, etc.)
% are located in separate files, for all files to be obtainable by main and subfiles,
% need to tranverse up to the root directory (../) then back to the appropriate files in /main
\usepackage{preamble}
\usepackage{mymacros}

% relative imports of images
% note that graphics path should be encased in {}, ends with / to define directory, 
% and finally separated by , without any lead or lag spaces.
% therefore graphicsspath should look like \graphiscpath{{abc/},{xyz/},{123/}}
% spaces in directory names are not recommended, however the the \usepackage[space]{grffile}
% will attempt to work with directories with spaces
\graphicspath{{./images/}}

%---------------------------------------------------------------------------%
%								Begin Document								%
%---------------------------------------------------------------------------%
\begin{document}

%------------------
% Table of Contents
%------------------
%!TEX root = ./main.tex

\begin{center}
	%-------------%
	% Institution %
	%-------------%
	{\LARGE\bf Massachusetts Institute of Technology \\
	\vspace{0.25\baselineskip}
	Department of Aeronautics and Astronautics}
	\vspace{\baselineskip}

	%----------%
	% Proposal %
	%----------%
	{\Large\bf Thesis Proposal \\
	\vspace{0.25\baselineskip}
	Doctor of Philosophy}
	\vspace{4\baselineskip}

	%--------------%
	% Thesis Title %
	%--------------%
	% {\Large\bf\underline{Title}:} \\
	\vspace{2\baselineskip}
	{\LARGE\bf An automated framework for robot design optimization and certification}
	\vspace{3\baselineskip}

	%--------------------%
	% Date of Submission %
	%--------------------%
	Date of Submission: \\
	\vspace{0.5\baselineskip}
	\today
	% September 13\textsuperscript{th}, 2016

	\vspace{8\baselineskip}

	\begin{tabular}{rlc}
		{\small \sc Author:}
	        	                    & Charles Dawson  & \\
	        	                    & PhD Candidate & \\
		\\ %space
		{\small \sc Committee:}
	        	                    & Chuchu Fan (Chair)  & \\
	            	                & TODO1 & \\
	            	                & TODO2 & \\
	            	                & TODO3 & \\
		\\ %space
		{\small \sc External Evaluator:}
	            	                & TODO4 & \\
	\end{tabular}
\end{center}


\newpage

%------------------
% Table of Contents
%------------------
\tableofcontents 
% some thesis documents wil also want a table of contents for figures and tables
% uncomment the commands below to create a table of contents for figures and tables
%\listoffigures 
%\listoftables
 
\newpage
 
% page numbering can be roman (e.g. i, ii, iii, iv), alpha (a, b, c, d)
% or arabic (1,2,3,4). Change the below option to roman for front matter text
% such as preface pages, alph for alpha lettering (e.g. Appendix maybe)
% or arabic for standard page numbering. If you want to restart the page 
% counter because you are using a new numbering format, you can use 
% \setcounter{page}{X} where X is the new number you want to count on.
\pagenumbering{arabic}

%!TEX root = ./main.tex

\begin{abstract}

    \noindent Before robots can be deployed in safety-critical environments, we must be able to verify they they will perform safely, ideally without the risk or expense of real-world testing. A wide variety of formal methods and simulation-driven techniques have been developed to conduct this verification, but they typically rely on difficult-to-construct mathematical models or else use sample-inneficient black-box optimization methods. In this thesis, I propose to develop a suite of tools that use program analysis tools like automatic differentiation to automatically construct mathematical models of the system under test and accelerate verification of robots and other autonomous systems. These tools rely on two technical innovations: first, the use of general-purpose automatic differentiation and probabilistic programming methods to introspect simulators of complex autonomous systems, and second: reframing the verification problem as a Bayesian inference problem (rather than an optimization problem) to make use of high-performance gradient-based inference algorithms. In addition to these technical innovations to solve verification problems, my thesis will also contribute a novel capability in the form of verification-guided design. Existing verification methods provide little insight to system designers about how to improve their systems to make them safer. In my thesis, I propose a novel adversarial inference algorithm to close the loop between verification and design, allowing the system designer to automatically generate and preemtively repair adversarial test cases to improve the safety of the system under test.

    % \noindent State the significance of the proposed research. Include long-term objectives and specific aims. Describe concisely the research design and methods for achieving these objectives. Highlight the specific
    % hypotheses to be tested, goals to be reached, or technology to be developed, which are intended to be
    % your original contributions. Avoid summaries of past accomplishments.

\end{abstract}

%!TEX root = ./main.tex

\section{Introduction}

% Introduce the need to manage complexity in designing/testing robot systems.

% Give a few of motivating examples where this might be helpful.

% Zoom in on design-analysis cycle

% Design tasks: exploring the design space. Fine-tuning designs.

% Analysis tasks: local adversarial testing, but also exploring diverse failure modes

% Closing the design/analysis gap: feeding failure modes back into the design process.

%!TEX root = ./main.tex

\section{Thesis Objectives}

This thesis aims to do X. In support of this goal, I will:

\begin{enumerate}
    \item Goal 1
    \item Goal 2
    \item Goal 3
    \item Goal 4
\end{enumerate}
  % what do I want to do
%!TEX root = ./main.tex

\section{Literature Review}
Background \& Significance section should be \textbf{3-5 pages}.

Sketch the background leading to the present research, critically evaluate existing knowledge, and
specifically identify the gaps that your research is intended to fill. State concisely the importance of the
research described in this proposal by relating the specific aims to the broad, long-term objectives.

% ADD YOUR TEXT HERE!!!!

%!TEX root = ./main.tex

\section{Expected Contributions}

Summarize gaps highlighted in literature review.

State how this thesis aims to fill those gaps.

Who will care? Why?
  % why will this be important?
%!TEX root = ./main.tex

\section{Approach: Design Optimization} \label{section:approach_design}

Along with the Objective \& Aims section, this is the most important part of the proposal. The majority of
your time should be spent making this part of your proposal strong, direct, and completely clear. Describe
the research design and the procedures to be used to accomplish the specific aims of the project; it is
generally most effective to do this according to the same outline as in the Objective \& Aims section. Include
how the data will be collected, analyzed, and interpreted. Describe any new methodology and its advantage
over existing methodologies. Discuss the potential difficulties and limitations of the proposed procedures
and alternative approaches to achieve the aims. As part of this section, provide a tentative timetable for the
project. Point out any procedures, situations or materials that may be hazardous and the precautions to be
exercised.

%!TEX root = ./main.tex

\section{Approach: Design Certification} \label{section:approach_test}

Along with the Objective \& Aims section, this is the most important part of the proposal. The majority of
your time should be spent making this part of your proposal strong, direct, and completely clear. Describe
the research design and the procedures to be used to accomplish the specific aims of the project; it is
generally most effective to do this according to the same outline as in the Objective \& Aims section. Include
how the data will be collected, analyzed, and interpreted. Describe any new methodology and its advantage
over existing methodologies. Discuss the potential difficulties and limitations of the proposed procedures
and alternative approaches to achieve the aims. As part of this section, provide a tentative timetable for the
project. Point out any procedures, situations or materials that may be hazardous and the precautions to be
exercised.

%!TEX root = ./main.tex

\section{Preliminary Results}

This section presents preliminary results from several of the research aims for this thesis, addressing both design and verification considerations. In particular, I have completed the following projects, which are described in more detail in the rest of this section.

\begin{enumerate}
    \item \textit{Certifiable Robot Design Optimization using Differentiable Programming} (accepted to RSS 2022). This work develops a design optimization framework based on automatic differentiation, paired with a statistical robustness analysis that allows the designer to verify the expected worst-case performance and sensitivity of the design. Sections~\ref{ch8:sec:design} and~\ref{ch8:sec:statistical_checking} provide more detail on this work.

    \item \textit{Robust Counterexample-guided Optimization for Planning from Differentiable Temporal Logic} (submitted to IROS 2022). This work explores the duality between design and verification through the lens of a two-player game, using automatic differentiation to find both locally optimal plans and adversarial counterxamples that guide further refinement of those plans. Sections~\ref{ch8:sec:adversarial} and~\ref{ch8:sec:game} provide more detail on this work.

    \item \textit{Automatic Failure Mode Discovery and Mitigation using Differentiable Programming} (TODO(fall 2022)). This work combines local optimization using automatic differentiation with a stochastic global search and clustering mechanism to automatically identify qualitatively different failure modes of an autonomous system. This results in a greater diversity of adversarial counterexamples that can be used to guide further design optimization. Sections~\ref{ch8:sec:adversarial} and~\ref{ch8:sec:game} provide more detail on this work.
\end{enumerate}

These works represent a progressively sophisticated exploration of the duality between design and verification, as represented in context of the goals of this thesis in Figure~\ref{ch8:fig:progression}. In the first work, I use a simple random sample of environmental parameters to guide the design optimization (analogous to the use of simple domain randomization in reinforcement learning TODO(citation)). In the second work, I use local optimization (enabled by automatic differentiation) to find adversarial examples, which accumulate into a dataset that guides the design optimization. In the third work, I expand the search for counterexamples from local optimization (which tends to get trapped in a single failure mode) to discover multiple qualitatively different failure modes, which provide a richer dataset for robust design optimization.

\begin{figure}[h]
    \centering
    \includegraphics[width=0.5\linewidth]{example-image-a}
    \caption{TODO: explain progression between these three works, and how they fit into overall thesis goals.}
    \label{ch8:fig:progression}
\end{figure}

\section{Glass-box design optimization}\label{ch8:sec:design}

TODO

\section{Glass-box adversarial testing}\label{ch8:sec:adversarial}

TODO

\section{Closing the design-verification loop}\label{ch8:sec:game}

TODO

\section{Statistical design verification}\label{ch8:sec:statistical_checking}

TODO
  % preliminary
%!TEX root = ./main.tex

\section{Milestones and Program Logistics}

\subsection{Classes and Degree Milestones}

Table~\ref{ch9:tab:course_requirements} shows my competed coursework, and Table~\ref{ch9:tab:degree_milestones} shows completed and anticipated degree milestones.

\begin{table}[h]
\centering
\caption{My completed coursework, satisfying all academic requirements for the doctoral program. Major: autonomy. Minor: controls.}
\label{ch9:tab:course_requirements}
\begin{tabular}{llll}
Semester    & Class                                             & Req.       & Status    \\ \hline
Fall 2019   & 16.413 Principles of Autonomy \& Decision Making  & major      & completed \\
Fall 2019   & 6.255 Optimization Methods                        & major/math & completed \\
Spring 2020 & 16.412 Cognitive Robotics                         & major      & completed \\
Spring 2020 & 6.832 Underactuated Robotics                      & minor      & completed \\
Fall 2020   & 18.385 Nonlinear Dynamics and Chaos               & minor/math & completed \\
Fall 2020   & 2.160 Identification, Estimation, and Learning    & minor      & completed \\
Spring 2021 & 16.S398 Formal Methods in Autonomy                & major      & completed \\
Fall 2021   & 6.843: Robotic Manipulation                       & major      & completed \\
Fall 2021   & 16.995 Doctoral Research \& Communication Seminar & RPC        & completed
\end{tabular}
\end{table}

\begin{table}[h]
\centering
\caption{Milestones towards my completion of the doctoral degree. Italicized milestones are anticipated.}
\label{ch9:tab:degree_milestones}
\begin{tabular}{rl}
Fall 2019 (September)          & Began studies at MIT             \\
Fall 2020 (December)           & Field evaluation complete        \\
Spring 2021 (May)              & Masters thesis submitted         \\
\textit{Fall 2022 (September)} & \textit{Committee meeting \#1}   \\
\textit{Fall 2022 (December)}  & \textit{Thesis proposal defense} \\
\textit{Spring 2023}           & \textit{Committee meeting \#2}   \\
\textit{Fall 2023}             & \textit{Committee meeting \#3}   \\
\textit{Spring 2024}           & \textit{Committee meeting \#4}   \\
\textit{Spring 2024}           & \textit{Thesis defense}
\end{tabular}
\end{table}

\subsection{Research Schedule}

My thesis research will proceed in stages, as outlined below.

Already completed:

Spring 2022
\begin{enumerate}
    \item Certifiable robot design optimization using differentiable programming
    \begin{enumerate}
        \item Develop design optimization tool using automatic differentiation
        \item Develop statistical robustness certification tool based on extremal types theorem
        \item Hardware deployment
        \item Accepted to RSS 2022
    \end{enumerate}
    \item Robust counterexample-guided optimization with temporal logic specifications
    \begin{enumerate}
        \item Define two-player zero-sum game between the designer and the verifier
        \item Incorporate counterexamples from the verifier to guide robust design optimization
        \item Use differentiable signal temporal logic for complex task specification
        \item Submitted to IROS 2022
    \end{enumerate}
\end{enumerate}

Future work

\textit{Fall 2022}
\begin{enumerate}
    \item Improving design optimization and verification through automated failure mode discovery
    \begin{enumerate}
        \item Add stochastic exploration to allow the verifier to find qualitatively different failure modes
        \item Use counterexamples from all failure modes to guide robust design optimization
    \end{enumerate}
\end{enumerate}

\textit{Spring 2023}
\begin{enumerate}
    \item Exploit structure in program state traces to automatically generate test suite with coverage guarantees
\end{enumerate}

\textit{Fall 2023}
\begin{enumerate}
    \item Extend automated failure mode discovery to design alternative discovery
    \item Write thesis
\end{enumerate}

\textit{Spring 2024}
\begin{enumerate}
    \item Write thesis
    \item Defend thesis and graduate
\end{enumerate}
  % future work and project planning

%---------------------------------------------------------------------------------------------%
% Bibliography
%---------------------------------------------------------------------------------------------%
\newpage

%-----------------------%
% automatic bib entries %
%-----------------------%
% enter your bibliographies using a .bib file
% for most formats, the unsrt argument (unsorted) will list the bibliographies as cited in the 
% text, rather than sorting them alphabetically. For the \bibliography entries, much like the 
% \graphicspath entries, each relative file directory is separated by a comma, no space,
% followed by the .bib file name, without the .bib extension
\bibliographystyle{unsrt} 
\bibliography{bib_example}
% example: \bibliography{../bib/bib_example,../bib/bib_example2,../bib/bib_example3}


%---------------------------------------------------------------------------------------------%
% Appendix
%---------------------------------------------------------------------------------------------%
\newpage
\appendix
\section{APPENDIX}

\subsection{SUB APPENDIX}



%---------------------------------------------------------------------------%
%								 End Document								%
%---------------------------------------------------------------------------%
\end{document}
