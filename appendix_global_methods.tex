\section{Appendix: Details on global methods}

\subsection{Details on power transmission case study}

The design parameters for this problem $x = (P_g, |V|_g, P_l, Q_l)$ include real power generation $P_g$ and AC voltage amplitude $|V|_g$ at each generator and real and reactive power demand $P_l$, $Q_l$ at each load (some loads, like vehicle charging, are capable of varying in response to network conditions). The exogenous parameters include the strength $y_i \in \R$ of each transmission line in the network; the admittance of each line is given by $\sigma(y_i) Y_{i, nom}$ where $\sigma$ is the sigmoid function and $Y_{i, nom}$ is the nominal admittance of the line. The prior probability for design parameters is uniform over the range of feasible generation and load behaviors, while the prior probability for exogenous parameters is Gaussian with a mean chosen so that the probability of the line admittance falling below 50\% of its nominal value is equal to the specified probability of failure ($5\%$ in our experiments).

The simulator $S$ solves the nonlinear AC power flow equations~\cite{dontiAdversariallyRobustLearning2021,dontiDC3LearningMethod2021} for the AC voltage amplitudes and phase angles $(|V|, \theta)$ and the net real and reactive power injections $(P, Q)$ at each bus (the behavior $\xi$ is the concatenation of these values). We follow the 2-step method described in~\cite{dontiDC3LearningMethod2021} where we first solve for the voltage and voltage angles at all buses by solving a system of nonlinear equations and then compute the reactive power injection from each generator and the power injection from the slack bus (representing the connection to the rest of the grid). The cost function $J$ is a combination of the generation cost implied by $P_g$ and a hinge loss penalty for violating constraints on acceptable voltages at each bus or exceeding the power generation limits of any generator, as specified in Eq.~\ref{eq:scopf_cost}. The data for each test case (minimum and maximum voltage and power limits, demand characteristics, generator costs, etc.) are loaded from the data files included in the MATPOWER software~\cite{zimmermanMATPOWERSteadyStateOperations2011}.